%remove hard-coded section headers
%proofread for clarity
%make sure formats correctly into course notes
%created 12/30/04 - thilmont

\section{General Table Information}

The layout of the contest table is shown in Figure
\ref{tablepic}. All measurements are approximate, although the only official
measurements are those of the actual tables.  The tables also have
seams, where sections of the table physically meet.  Make sure your
robot is capable of facing the imperfections of the board.

Also, do not rely on the texture of the contest table surface. The
tables will be steadily worn down over IAP, and some sections may
repainted before the final contest.  In addition, the tables were made
during the summer, and therefore their surfaces may have warped
slightly by the time of the contest.  A properly designed robot should
be able to surmount these problems.
 
\section{Table Description}

The table is flat. The entire 6' X 8' area is surrounded by a 2'' wall
on all four sides.  The surface of the table is white, but there are a
number of 2'' wide black lines that can be used for navigation.  For
the lines running along the walls of the table, there are 6'' between
the center of the line and the wall.  The lines running north-to-south
across the middle of the table have their centers at 1'6'', 3' and 4'6'' from the 8' walls.
The center line goes through the middle of the two starting areas, the white area being represented by the black 18''x18'' cross and the black area by the 18''x18'' white cross.

On the two 8' sides of the table there are slots 2' long and 1.5'' in height.

There are two vertical obstacles on the board, represented on the
figure by solid rectangles.  They are each 2'' x 12'' at the base, and
stand 6'' tall.  They are fixed and your robot will not be able to
move or destroy them.  They are not tall enough to obstruct IR beacon
signals.

There are 24 small, approximately 1'' diameter, balls on the table.
They are arranged in groups of four and represent groups of frightened
freshmen. The small balls are placed at the corners of an 8'' x 8'' square.

There are also 2 larger, 3'' diameter balls resting along the
north-to-south centerline of the table.  These 2 balls are placed
2'6'' from the walls. These represent the particularly cool freshmen
that every living group wants in their house.

There are shallow divots on the table for all of the balls to
rest on.  Balls may be pushed around, picked up, knocked off the
table, or otherwise moved by the robot.

Robots begin in opposite corners of the board in a 12'' x 12'' area
with a start light in the center.  The presence/lack of the solid 6''
x 6'' black square in each starting area assists the robot in
determining which side it is on. 

\epsfysize=5.0in
\begin{figure}[htp]
\begin{center}
\epsfbox{contest/table2004.eps}
\caption{2005 Contest Table Design}
\label{tablepic}
\end{center}
\end{figure}


