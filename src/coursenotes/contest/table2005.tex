%remove hard-coded section headers
%proofread for clarity
%make sure formats correctly into course notes
%created 12/30/04 - thilmont

\subsection{General Table Information}

The layout of the contest table is shown in Figure
\ref{tablepic}. All measurements are approximate, although the only official
measurements are those of the actual tables.  The tables also have
seams, where sections of the table physically meet.  Make sure your
robot is capable of facing the imperfections of the board.

Also, do not rely on the texture of the contest table surface. The
tables will be steadily worn down over IAP, and some sections may
repainted before the final contest.  In addition, the tables were made
during the summer, and therefore their surfaces may have warped
slightly by the time of the contest.  A properly designed robot should
be able to surmount these problems.
 
\subsection{Table Description}

The table is flat. The entire 6' x 8' area is surrounded by a 2'' wall
on all four sides.  The surface of the table is white, but there are a
number of 2'' wide black lines that can be used for navigation.  For
the lines running east-west along  the table, there are 18'' between
the center of the line and the wall.  The lines running north-to-south
across the middle of the table have their centers at 6'', 3' and 5'4'' from the 8' walls.
The center line goes through the middle of the two starting areas

On the two 8' sides of the table there are slots 2' long and 1.5'' in height.

There is one vertical obstacle on the board, represented on the
figure by the blue rectangle.  It is 2''x18'' and
stands 10'' tall.  This obstacle is fixed and your robot will not be able to
move or destroy it.  

There are 44 (22 red and 22 green) approximately 1'' diameter balls on the table.
They are arranged in groups of four and represent groups of Gedi Council Voters. The balls are placed at the corners of an 8'' x 8'' square. There is a set of balls in the center of the table, stradling the barrier. Each slot on the side of the table has a set of balls on eaither side of it. The center scoring areas each have a set of balls in front of them. Finally the corner scoring areas each have a set of balls oppisit the outer corner.

There are shallow divots on the table for all of the balls to
rest on.  Balls may be pushed around, picked up, knocked off the
table, or otherwise moved by the robot.

Robots begin in the center of the table, with the white robot starting on the black cross and the black robot starting on the white cross.  The presence/lack of the solid 8''x 8'' black squares in each starting area assists the robot in
determining which side it is on. 

\epsfysize=5.0in
\begin{figure}[htp]
\begin{center}
\epsfbox{contest/table2005.eps}
\caption{2005 Contest Table Design}
\label{tablepic}
\end{center}
\end{figure}


