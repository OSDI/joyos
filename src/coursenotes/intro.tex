%% fix assignment dates
%% updated 1/4/04, tdwan

\chapter{Introduction to 6.270}
\pagenumbering{arabic}
\begin{quote}
6.270 is a hands-on, learn-by-doing course in which participants design and
build a robot that will play in a competition at the end of IAP. Each team
begins with a box of components from which members must produce a robot that can
manipulate game objects on a playing field inhabited by an opponent. Unlike the
machines in 2.007 Introduction to Design (formerly 2.70), robots in 6.270 are
completely autonomous.  Human intervention during the round is forbidden.

The goal of 6.270 is to teach students about robotic design by giving them the
hardware, software, and information they need to design, build, and program
their own robot. The concepts and applications taught in this class are related
to various other MIT classes (e.g. 6.01, 6.002, 6.004 and 2.007). However there are
no formal prerequisites for 6.270. Students with little or no experience will
find that they will learn everything they need to know from working with each
other, attending lectures and workshops, and hacking on their robots.

6.270 is an extremely challenging course and requires participants to be willing to
put in a real effort. Most students will spend in excess of one hundred hours
building their robots. If you are willing to commit the time and energy needed
for this class, you will have a great time and even learn something along the
way.

So, prepare yourself for three and a half weeks of immersion into the world of
robotics. Welcome to 6.270!
\end{quote}


\section{Staff}

The 6.270 staff is composed of volunteers chosen from course alumni.  You should
feel free to approach them for help or with any questions you might
have. The staff consists of two groups of people: Organizers and Teaching
Assistants.

The Organizers are the people responsible for running the course. In addition to
teaching and staffing the lab, they handle all the administrative duties, such
as speaking with sponsors, ordering parts, defining the contest, and ensuring
everything runs smoothly. A course the size of 6.270 requires a large amount of
work and planning, and the Organizers have spent over a year preparing for this
competition.

The Teaching Assistants (TAs) are selected by the Organizers to assist in
teaching the course. They work primarily during IAP and their job description
requires that they help teach recitations, staff the lab, and build
demonstration robots and placebos. They are often also called upon by the
Organizers to assist in certain tasks.

Members of the staff and their emails are listed in Figure \ref{07stafflist}.

Organizers and TAs receive very little compensation for the work they do.  They
are here only because they love 6.270 and want others to have the same
opportunity to enjoy it as they did. In return, the staff asks only that you put
in the time and effort necessary to get as much as possible out of the course.
If you enjoy your experience in this course and would like to see it continue to
be offered, please consider joining the staff in future years. It is only
through the continued enthusiasm and selflessness of course alumni that 6.270 is
able to remain the most popular student-run activity at MIT.

\begin{figure}[htbp]
 \begin{center}
 \begin{tabular}{|c|c||c|c|}
  \hline
  
%  Jessi Jesse Michael Steven Peter Evan Mike

%  Everest Rachel Sean Steven Sam Amber

  \bf{Organizer}&   \bf{Email}& 					\bf{TA}&        	\bf{Email} \\
  \hline \hline
  Jessi Ambrose& 			\tt{jambrose}&      		Samuel Evans& 		  \tt{s\_evans}\\
  Steven Herbst& 		\tt{herbst}&     			Amber Hess& 		  \tt{avihess}\\
  Peter Iannucci&		\tt{iannucci}&				Everest Huang&       	\tt{everest}\\
  Evan Iwerks& 		\tt{iwerks}&  					Steven Ji&      	\tt{zyji}\\
  Mike McCanna&				\tt{acrefoot}&        		Rachel Meyer& 		\tt{remeyer}\\
  Jesse Moeller&   \tt{jmoeller}&					Sean Wahl& 		\tt{swahl}\\
  Michael Morris-Pearce& 	\tt{mikemp} & & \\		
  \hline
 \end{tabular}
 \end{center}
 \caption{2009 6.270 Staff and email list}
 \label{07stafflist}
\end{figure}

As we compile it, we will be placing more information, including pictures, lab
hours, and skill lists (who to contact for help with a certain aspect of the
contest) on the web:

\begin{quote}
\begin{verbatim}
http://www.mit.edu/~6.270/staff/
\end{verbatim}
\end{quote}

\section{Kits and Tools}

The 6.270 kit, valued at about \$1500, is yours to keep at the end of the
contest. This is made possible by financial support from the EECS department and
the course's commercial sponsors. If your team does not present a robot to the
Organizers at the qualifying round of the competition, or if you are asked to
leave the course, you will be required to forfeit the kit back to the EECS
department.  Teams who do not return the entire kit when asked will be charged
the full \$1500 through the Bursar's office.

There are tools available for in-lab electronics work, but these resources will
probably be over-burdened, especially towards the end of IAP.  Therefore, in
addition to the kit, a set of tools will be reserved for purchase by your team.

We offer the tools at a cost that is a lower bulk price for us. However, if
you're looking to find tools on your own, you'll want to get the following:

\begin{itemize}
\item Soldering iron
\item Soldering stand
\item Wire cutters
\item Needle-nose pliers
\item Helping hands
\item Diagonal cutters
\end{itemize}

\section{Electronic Communication}

\subsection{Mailing Lists}

Participants are encouraged to check their email daily, since notices are sent
out often and without warning.

\begin{itemize}

\item {\bf 6.270-staff@mit.edu} is the main adminstrative list
for 6.270. All questions or comments concerning the course should be
directed to this list. Current staff as well as all past Organizers
are members of this list and will be able to help answer your
questions.

\item {\bf 6.270-rules@mit.edu} is for
question about the rules, contest proceedings, or the validity of your
robot. The Rules Committee is comprised of a subset of the
Organizers.

\item {\bf 6.270-contestants@mit.edu} is the primary announcement
list for the course. You will be added automatically and all important
information for participants will be sent to this list.

\item {\bf 6.270-fanclub@mit.edu} contains all previous participants in 6.270,
as well as some other members of the MIT community. This is an extremely
low-traffic list that you will be added to as a 6.270 alumnus. It is used only
for general interest 6.270 announcements, and under no circumstances should
anyone but the Organizers send email to it.

\end{itemize}

\section{Laboratory Facilities}

During the course of constructing your robot, you will have access to
workspaces, tools, and computers in the following areas:

\subsection{6th Floor Laboratory}

The 6th floor lab (Room 38-600) is the center of activity for the
course. This lab is supervised by the 6.270 staff, and other teams
will be present to share ideas with. Among the useful facilities in
this lab are workbenches for building your robot, computers for
programming, and two contest tables for testing.

The lab will be open and staffed from 11 am to 11:30 pm on weekdays and
noon to 10 pm on weekends. During the final few days of the course,
the lab may be open 24 hours a day. If you need to call the lab, the
phone number is x3-7350, but please do not place or receive personal
calls too often. The phone line needs to be kept available for
official use, and the staff is too busy to run a personal messaging
service.

Since this lab is on loan to 6.270 by the EECS department, please
be courteous. Do not touch equipment not explicitly designated for 
6.270 use and treat all lab staff with respect.
Be aware that the lab will be closed at night for the first three weeks 
of the competition. So be ready to leave the lab when it's time to close. 
We reserve the right to penalize you for offenses related to the 
abuses of the lab or its staff

\subsection{Other Facilities}

Since the course software is available for a number of computer platforms, some
students choose also to program their robots from their own computers. Teams
with access to laptops may find this option especially useful even when working
in the lab, since it frees them from waiting for the lab computers.

\subsection{Etiquette}

When working in the lab or at Athena, you will be expected to be respectful to
those around you. We expect you to obey the following rules at all times:

\begin{enumerate}

\item{\it Noise.}  Your robot will be quite noisy. When working at Athena,
please minimize the operation of your robot. If others are disturbed by the
noise, stop running the robot or move to another cluster.

\item{\it Hardware.}  Do not solder, cut, or glue any hardware in the clusters
or around the computers in lab. Debris can get lodged in the keyboards and
damage the computer. Furthermore, when working on the lab benches with solder or
glue make sure you are working on top of a piece of cardboard
area to prevent damage to the tables. Failure to do so may result in the loss of 
lab privileges.

\item{\it Tidiness.}  Do not take up more space than you need. Be tidy. The lab will
be crowded and people will need places to work. Your team should try to limit the
area that it uses to two workbenches, or one if the lab is very crowded.

\item{\it Multiple Machines.}  Do not log on at multiple machines.
When the lab is crowded, please try also to minimize the number of
people on your team who are logged on. The lab does not have enough
computers to support everyone being logged on at once.

\end{enumerate}

Violations of the rules of etiquette will not be met with severe penalty.
Causing a disturbance in the Athena clusters will result in your team's forced 
withdrawal from the competition, and your kit will be confiscated. Repeated 
violations of lab etiquette will be investigated by the 6.270 staff, and will
be dealt with on a case by case basis. Penalizing teams is unpleasant for both
the staff and competitors, so please respect one another and follow the
etiquette.

\section{Credit Guidelines}

6.270 is offered as MIT subject 6.185 for 6 units of Pass/No Record credit with
the further option to receive 6 Engineering Design Points (EDPs). Taking the
course for credit is optional, but we encourage your full input.
Receiving credit will give you formal recognition on your transcript in addition
to the academic credit.

It is the job of the instructors to ensure that credit is properly awarded to
students deserving of it. In order to properly evaluate your performance, it is
necessary that you report your work. The credit requirements are structured to
allow your instructor to authorize credit and also assist you in the learning
process.

The following guidelines must be completed in order to receive 6 units of
academic credit, and if desired, 6 EDPs:

\begin{itemize}

\item {\bf Robot Web Page.}
We will describe what is required on the web page at the end of the course. You
don't need to worry about this right now.

\item {\bf Assignment Completion.}
Seven assignments will be handed out during the first two weeks of the course.
These assignments were made to help guide participants in making effective and
competitive robots. Participants wanting credit are expected to complete the
assignments on-time. Failure to complete the assignments on time will result in
gradual penalties ultimatly resulting in forfeiture of the competition.

\item {\bf Completed Robot.}
The team must ``show'' a robot at the qualifying round. Its functionality, or
lack thereof, has no effect on the team's members receiving credit for the work
they have done.
\end{itemize}


{\bf If you need extra time on
assignments 1 through 4, please talk to an organizer---we are generally
understanding. However under no circumstances will extensions be granted for
Assignment 5 (impounding). For your own benefit (and ours), please avoid this
and start assignments early.}


\section{Schedule}

\begin{itemize}
\item {\bf General Lectures.}  Lectures will be held during the first week of
the course to introduce you to the basics of robotics. These lectures are meant
to provide you with an overview of the information necessary to create a working
6.270 robot.

\item {\bf Laboratory Sessions.}  Staff members will be present in the 6th floor
lab to assist you in the construction of your robot.  One of the goals of 6.270
is to encourage interaction, and the lab is a great place to share ideas with
others and experiment with new ideas.

\item {\bf Workshops.}  During the first two weeks of the course, workshops will
be taught by experienced staff members. These workshops will cover many aspects
of the course, from soldering to programming to construction. They are planned
to help reinforce the material learned in lecture.

\end{itemize}
