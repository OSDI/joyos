\section{The Competition}
\label{compete}

The competition will be a double elimination tournament.  Robots compete
head to head in successive rounds until they lose twice.  When all but one
robot has been eliminated, that robot will be crowned champion.

In each round, if there are an odd number of robots, the robot that
has scored the most cumulative points will face a placebo.  After that
robot is paired, the robot with the next highest cumulative score is
paired with the robot with the lowest cumulative score, and so forth.
Ties are resolved randomly.  Note that this system is only used for
deciding what opponent each robot will face -- the order in the round
and the side of the table are still randomly selected.

All rules are subject to change at any point during competition, at
the discretion of the Organizers.  This power will be used sparingly.

\begin{itemize}

\item {\bf Contest, Qualifying Rounds.}
The first two rounds of the contest serve as seeding rounds.  If
a robot demonstrates the ability to score points, regardless of whether it wins
or loses, it will be allowed to proceed to the competition rounds on Thursday.
If it does not, modifications may be made, and it may attempt to qualify in lab
against a non-competitive placebo.  If it cannot score points against the
placebo, it will not qualify for the rest of the contest.  Robots may qualify
until impounding.

For the sake of efficiency, we will be publishing a schedule for the round
matchups as soon as one is available.  We promise that each robot will not be
asked to compete until the time that is posted.  If a team is not present when
we are ready to start the match, the team automatically forfeits.

\item {\bf Contest, Final Rounds.}  This is the main competition that
  everyone comes to see.  Robots will compete until all but the winner
  have been eliminated.  Once two teams remain in the contest, all
  previous records will be erased and three rounds will be run.  The
  team with more wins in the three rounds wins the competition.

\end{itemize}

\section{Rules}

The following rules of play are meant to ensure a fair and interesting contest.
Contestants are responsible for knowing and following these rules. If you have
any questions or doubts about the legality of your robot, please ask the Rules
Committee by emailing \verb^6.270-rules@mit.edu^ for an official ruling.

\subsection{Period of Play}
\begin{enumerate}


\item The contestants will have a 60 second calibration period to setup their
robot. A team can use this time to tune sensors to the table and enter the
goal points for the match. A team {\bf cannot} use this time to enter any of
the following information:

\begin{enumerate}
\item Starting orientation
\item Starting color/side
\item Specific strategy
\item Opposing team
\end{enumerate}

Your robot must dynamically sense and make decisions during the round.

\item By the end of the calibration period, teams must place their robots in
the assigned starting zone and orient in the assigned direction. The direction
will be one of four options: North, South, East, and West. The side marked
``front'' must point in this direction.

\item Robots may not supply power to their actuators at this point.  If a robot
does, it has false-started. Also, if a team takes more than the alotted 60
seconds for setup, this counts as a false start.  If a robot false-starts twice,
it forfeits that match.

\item The robots will recieve a radio signal informing them to start the match.
After the start signal,the robots may turn on any motors or actuators and the
robots have 90 seconds to compete and score points. Software to detect the
signal will be provided.

\item To start the match, contestants will be directed to press the ``Go''
button on their robot. This is the only time contestants are allowed to touch
their robot after calibration.

\item During the match, the contestants must stand back from the table. Any
contestant who touches the machines or otherwise interferes with the match will
cause his machine to forfeit the match. All robots must be controlled solely by
their onboard computer.

\item At the end of 90 seconds, the robot must turn off electrical power to its
actuators.  Any robot that fails to shutdown forfeits that match. Software is
provided to do this.

\item The match ends when all robots and game objects on the table come to a
rest.

\item If one of the robots forfeits the match for any reason, it will be
replaced with a placebo and the match will continue.

\end{enumerate}

\subsection{Kits}
\begin{enumerate}

\item Some parts in the kit are considered tools and may not be used on the
robot.

\item Robots must be built only from the parts in the kit, except when
explicitly allowed by other rules.

\item Teams may trade only functionally identical components. This includes
trading identical LEGO parts of different colors and replacing broken
components. Any other trading is forbidden.

\end{enumerate}

\subsection{Robots}
\begin{enumerate}

\item The robot structure must fit within a one foot cube at the start of a
match; however, they may expand once the match has begun.  Wires may be
compressed, if necessary, to fit this measurement.

\item All parts of a robot must be connected via LEGO. Robots may not separate
or have a tendency to break into multiple parts.

\item Decorations may be added to a robot provided they perform only an
aesthetic function, and not a structural one.

\item Robots may not intentionally damage, or attempt to damage, the opponent
robot or its microprocessor board.

\item No parts or substances may be deliberately dropped onto the playing field.

\item Any robot that appears to be a safety hazard will be disqualified from the
competition.

\end{enumerate}

\subsection{LEGO}
\begin{enumerate}

\item Only LEGO parts may be used as robot structure.

\item A robot's structure may not be altered after impounding. Repairs may be
made between rounds if time permits.

\item LEGO pieces may not be modified in any way, with the following exceptions:

\begin{itemize}

\item The LEGO baseplate may be modified freely.

\item LEGO pieces may be modified to facilitate the mounting of sensors and
actuators. However, such modification cannot be structural.

\item LEGO pieces may be modified to perform functions directly related to the
operation of a sensor. For example, holes may be drilled in a LEGO wheel to help
make an optical shaft encoder. Such modifications cannot be structural.

\end{itemize}

\item LEGO pieces may not be joined by adhesive.

\item Lubricants of any kind are not permitted.

\item Rubber band or tape may be applied to LEGO wheels and treads to alter the
coefficient of friction.  See Section \ref{non-lego} for more details on the use
of rubber bands.

\item Wheels may be stuffed with any material within reason.  Students usually
stuff them with rubber bands, LEGO, or hot glue.  Double-check with an Organizer
before using another material.

\end{enumerate}

\subsection{Software}
\begin{enumerate}

\item A robot's program cannot be altered after impounding.

\item In the event of a memory failure, a copy of the robot's program may be
downloaded from an official computer between rounds. The program available for
download will be the version submitted on impounding. Contestants are not
permitted to download any code to the robot from their own computer or any other
computer.

\item A robot may not be told its position or be given information about its
opponent. It may only deduce this information after the match has begun.

\end{enumerate}

\subsection{Non-LEGO parts}
\label{non-lego}
\begin{enumerate}

\item Sensors, actuators, and other Non-LEGO parts may not be used as structural
components.

\item Non-LEGO parts may be attached to no more than five LEGO parts.

\item Non-LEGO parts may be freely modified to assist in their operation.

\item A reasonable amount of cardboard, other paper products, and tape may be
used for the purpose of creating optical shields for sensors.

\item Wire may only be used for electrical purposes and may not be dragged on
the playing surface. Wires that extend outside of the robot should be tied back.

\item String may be used to convey force between moving parts (i.e. pulley
systems) but may not be used for structural support. String may also be used to
stuff tires for stiffening purposes.

\item Extraneous components may not be added to a robot for the purpose of
adding weight.

\item Rubber bands cannot be used for structural support.  Only thin rubber
bands that are supplied by the organizers (\#16 and \#32) may be used.  Rubber
bands may be used only for the following purposes:

\begin{itemize} 

\item Rubber bands may be used to store energy to affect the motion of moving
parts. Rubber bands used for this purpose must be touching at least one piece of
LEGO. LEGO pieces connected by a single rubber band or a chain of two rubber
bands must move relative to each other.

\item Rubber bands may be used to stuff tires for stiffening purposes.

\item Rubber bands may be used to add strength to tread between chain.

\item If you would like to use rubber bands for another purpose, please make
sure you check with Rules Committee first.

\end{itemize}
\end{enumerate}

\subsection{Placebos}

Placebos are staff-built ``demo'' robots, which should not present
significant competition to a well-built entry. In matches involving
only one robot player, a placebo will stand in for the other.  Teams
should consider a match against a placebo to be just like a match
against any other robot. The placebo will conform to all rules.

\section{Extra Electronics}

Each team is given two pools of resources from which they can obtain more
sensors and motors to their robots---20 sensor points and a 30 dollar allotment.
A team's robot is disqualified if the total points of the sensors on the robot
after impounding exceeds the sum of 20 points' worth of sensors and the basic
set of sensors originally given in the kit, or if the electronics monetary value
exceeds 30 dollars.

\subsection{Electronic Modifications}

Each team is free to modify any of the electronics or actuators. However, using
electronics in a non-standard fashion is a risk that your team must consider
before making any modifications. If electronics (including the Handy Board and
expansion board) or actuators are broken because of these non-standard
modifications, the team will not be given replacement parts. Any additional
parts used for modifications count towards the 20 sensor points and 30 dollar
limit.

Before making any modifications, each team {\em must} consult an Organizer.
Before making the modification, the team must turn in a list of all the parts
(kit or non-kit) used for the modification. The team must also turn in a design
report that includes a description of the modification, a schematic of all added
circuitry. This design report must be turned in {\em before} the modification is
made on the robot. 

All modifications made by all teams will be posted online once the design report
is given the Organizers.  Any questions or concerns about a potential
modification must be emailed to
\begin{verbatim}6.270-rules@mit.edu\end{verbatim}. 

\subsection{The Sensor Store}

In order to encourage variety in robot designs, each team will be
given a sample set of sensors and an allowance of 20 sensor points
with which they may obtain additional sensors from the Organizers. No
refunds will be permitted, so contestants are encouraged to experiment
with the sample sensors before making decisions on which sensors to
get. Note that sensors purchased from the sensor store are considered
kit parts and must be used in accordance with all applicable rules.

Sensors can be traded as long as the following rules are observed:

\begin{enumerate}

 \item Teams are allowed to trade sensors of equal point value with
 other teams.

 \item A team can trade a sensor with the Sensor Store as long as the
 sensor is returned in original condition.

 \item A broken sensor may be traded in for a new sensor of the same
 type at a cost of 5 dollars or the at-cost price of the sensor
 rounded up to the nearest dollar, whichever is highest.  Paying for a
 broken sensor this way does not count to the 30 dollar nor the 20
 sensor point allotment.
\end{enumerate}

\subsection{30 Dollar Electronics Rule}

A team may spend up to 30 dollars of its own funds to purchase
electronic components from non-6.270 sources. This is not to be
confused with the sensor points, which can only be used for the Sensor
Store. Contestants are encouraged to use this rule to explore new ways
of sensing or otherwise make their robot more interesting. Teams
taking advantage of this provision, however, must abide by the
following guidelines:

\begin{enumerate}

 \item Each team is required to submit receipts for every additional
 component purchased. All receipts must be submitted at impounding.

 \item If a team wishes to use parts obtained through means other than
 retail purchase, an equivalent cost will be assigned by the
 Organizers. This estimate must be obtained in writing from the
 Organizers.

 \item Resistors rated less than 1 watt and capacitors valued less
 than 100 \u F may be used freely, without counting towards the 30 dollar
 total.

 \item Extra servos can be purchased from 6.270 staff at varying prices based on quality.

 \item If a team needs a replacement servo, the team must pay for the
 servo at retail price rounded up to the nearest dollar.  The
 replacement fee does not count towards the 30 dollar allotment.

 \item Only components on the actual impounded robot will count
 towards the 30 dollar allotment.

\end{enumerate}
