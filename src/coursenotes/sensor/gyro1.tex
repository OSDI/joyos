Analog Devices and Intempco provide surface-micromachined integrated circuit gyroscopes mounted 
on Happyboard-compatible printed-circuit boards. The gyro can be used to measure your robot's 
rate of turn or (with some numerical integration) angular orientation, which can be a signicant
aid to navigation. These devices integrate the mechanical parts of the gyro along with the
necessary circuitry on a single chip, so they are very small (about 7mm square). To make them
easier to handle and use, the gyros are supplied on small printed-circuit boards
containing all necessary external components to which you solder pins.  The pins plug directly
into the Happyboard sensor input connector.

Your gyro board may have one of a few different gyros.  They will be one of the ADXRS300, ADXRS610,
ADXRS622, or AD22307.  The most significant difference between them is the full-scale range (or
sensitivity), which will be $250\,^{\circ}\mathrm{/s}$ or $300\,^{\circ}\mathrm{/s}$, or $5-7\mathrm{mV/}\,^{\circ}\mathrm{/s}$.  The data sheet for your gyro is 
available at www.analog.com.  The AD22307 is a specially-branded version of the ADXRS622. 

A gyro measures rate of turn.  Our gyros operate on a single 5V power supply, as supplied by the
Happyboard's analog input connectors.  They produce a single voltage output in the range of 0 to 5V,
so they are wired and used like most other three-wire sensors. They are not, however, ratiometric to
the power supply like accelerometers or potentiometers. The output of the gyro when
it is not rotating is nominally 2.5V, midscale on the Happyboard's A/D converter. The
nominal sensitivity of the gyro is 5-7mV per degree per second of rotation, depending on which
gyro you have. You should calibrate your gyro's sensitivity, and make sure that the robot never
turns faster than the full-scale range.  Calibrate the gyro's offset at the beginning of each run.

Some gyro boards may say 'MIT Noisy Gyro' on them.  This doesn't mean that the gyro is noisy; rather 
that noise is explicitly injected into the signal path to improve the effective resolution of the A/D 
converter.  This was more important on the older 6.270 controllers which had an
8-bit A/D converter, but has no detrimental effect when used with the Happyboard.

To get from rate of turn to swept angle, it is necessary to take a series of readings 
of the gyro's output and do a simple numerical integration. Techniques and code for 
performing the integration will be presented in lecture and in a handout.

Figure \ref{gyro1} shows the mechanical configuration of the gyro board and the
necessary connections.  To operate properly, the gyro board should be
mounted in an approximately horizontal position (parallel to the surface of the earth), 
assuming that you intend to measure the turn angle (yaw) of your robot.  It can be
plugged directly into the Happyboard sensor input connector, or it can be fastened in place 
with double-sided-sticky tape elsewhere on your robot.

%Contacts (all at Analog Devices):

%\noindent Jack Memishian \verb^(john.memishian@analog.com)^\\
%Mark Nelson \verb^(mark.nelson@analog.com)^\\
%Howard Samuels \verb^(howard.samuels@analog.com)^\\

\epsfysize = 2.0in
\begin{figure}[htp]
\begin{center}
\epsfbox{sensor/gyro1.eps}
\caption{Mechanical configuration of the gyro board.}
\label{gyro1}
\end{center}
\end{figure}
