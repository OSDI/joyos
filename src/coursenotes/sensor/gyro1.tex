For this year's competition, Analog Devices is making available surface-micromachined
integrated circuit gyroscopes which can be used to measure your robot's rate of turn or
(with some numerical integration) angle of turn, which can be a signicant aid to navigation.
These devices integrate the mechanical parts of the gyro along with the necessary
circuitry on a single chip, so they are very small (about 7mm square). To make them
easier to handle and use, the gyros are supplied on 1.25" square printed-circuit boards
containing all necessary external components and a pin header for connections.
A complete data sheet for the gyros (ADXRS300) can be found on the Analog Devices
website, www.analog.com (select iMEMs Gyroscopes under the MEMs Technology
category). However, most or all of what you need to know to use the gyros successfully
in your robot will be presented here and in a lecture during the First week of the course.

A gyro measures rate of turn, and these gyros have a full-scale range of +/-300 degrees
per second. They operate on a single 5V power supply, as supplied by the Handyboard's
I/O channels, and produce a single voltage output in the range of 0 to 5V, so they are
wired and used like most other three-wire sensors (they are not, however, ratiometric to
the power supply like accelerometers or potentiometers). The output of the gyro when
it is not rotating is nominally 2.5V, midscale on the Handyboard's A/D converter. The
nominal sensitivity of the gyro is 5mV per degree per second of rotation. (Calibration
routines may be used to take out the gyro's offset and sensitivity errors at the beginning
of a run.)

Since the A/D converter converts its whole 0 to 5V input range into 256 total codes,
its resolution is only a bit better than 20mV. This means that the A/D can only resolve
the output of the gyro to within about 4 degrees per second. To improve the resolution of
gyro, Analog has designed and built a specialised board for 6.270. This board introduces 
some noise into the gyro output voltage, which allows us to "dither" the reading,
increasing the effective resolution of the gyro.

To get from rate of turn to swept angle, it is necessary to take a series of readings 
of the gyro's output and do a simple numerical integration. Techniques and code for 
performing the integration will be presented in lecture and in a handout.

Figure \ref{gyro1} shows the mechanical configuration of the gyro board and the
necessary connections.  To operate properly, the gyro board should be
mounted in an approximately horizontal position, assuming that you
intend to measure the turn angle (yaw) of your robot.  It can be
fastened in place with double-sided-sticky tape.

%Contacts (all at Analog Devices):

%\noindent Jack Memishian \verb^(john.memishian@analog.com)^\\
%Mark Nelson \verb^(mark.nelson@analog.com)^\\
%Howard Samuels \verb^(howard.samuels@analog.com)^\\

\epsfysize = 2.0in
\begin{figure}[htp]
\begin{center}
\epsfbox{sensor/gyro1.eps}
\caption{Mechanical configuration of the gyro board.}
\label{gyro1}
\end{center}
\end{figure}
